\section{Analysis and Results}

The circuit that LTSpice simulates is redrawn in Figure~\ref{fig:sig_circuit}
for analysis.

\begin{figure*}[h]
\begin{circuitikz}[]
    \draw (0,0) to[vsourcesin, name=vs] (0,6);
    \node [below left, align=center, inner sep=12pt] at (vs.e)
    {$0-3.3$\unit{\volt}\\$90$\unit{\hertz} PWM};
    \draw (0,6) to [R=$R$] (2,6) to [C=$C$] (2,0) to node[ground]{} (2,0);
    \node [ground] at (0,0) {};

    \node [op amp, noinv input up](A1) at (4.5, 5.52) {\texttt{OP292}};
    \draw (A1.+) to[short] (2,6);
    \draw[-latex] (A1.up) -- ++(0,0.5) node [above] {$V_+$};
    \draw (A1.down) -- ++(0,-0.25) node[ground] {};
    \draw (A1.out) to[short, *-] ++(0, -2.0) -- ++(-2.39, 0) to (A1.-);

    \draw (A1.out) to[short] ++(0.5,0) to [R=$R$] (7.5,5.52) to[short] ++(0.5,0)
    to [C=$C$] (8,0) to [ground] (8, 0);
    \node [ground] at (8.0, 0) {};

    \node [op amp, noinv input up](A2) at (10.5, 5.04) {\texttt{OP292}};
    \draw (A2.+) to[short] (8,5.52);
    \draw[-latex] (A2.up) -- ++(0,0.5) node [above] {$V_+$};
    \draw (A2.down) -- ++(0,-0.25) node[ground] {};
    \draw (A2.-) -- ++(0,-2.0) coordinate(FB) to [R=$R_{in}$] (9.35, 0)
    coordinate (tmp) node[ground]{} (FB) to [R=$R_f$, *-] (FB -| A2.out) -- 
    (A2.out);

    \draw ($ (FB) + (2.38, 0) $) to [R=$R_{\text{load}}$] ($ (tmp) + (2.38, 0)
    $) node[ground]{};

    \draw ($ (FB) + (2.38, 0) $) to[short, *-o] ++(1,0) node[above]{$v_o$};
\end{circuitikz}
\caption{The signal generator circuit.}
\label{fig:sig_circuit}
\end{figure*}

The $0-3.3$\unit{\volt} PWM signal is filtered to extract the modulated sine
wave with a the first $RC$ combination.
