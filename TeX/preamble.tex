%\usepackage{lgrind}        % convert program listings to a form includable in a LaTeX document
% \usepackage[export]{adjustbox}
\usepackage{amsmath}
\usepackage{amssymb}
\usepackage{amsthm}
\usepackage{xcolor}
\usepackage{caption}
\captionsetup[figure]{font=small}
\usepackage{chapterbib}    % allows a bibliography for each chapter (each labguide has it's own)
\usepackage[siunitx, RPvoltages]{circuitikz}[american]
\usepackage{enumitem}
\usepackage{epsfig}
\usepackage{epstopdf}
% \usepackage[top=1in, bottom=1in, left=1in, right=1in]{geometry}
\usepackage{graphics}      % standard graphics specifications
\usepackage[]{graphicx}      % alternative graphics specifications
\graphicspath{ {./figures/} }
\usepackage{lipsum}
% \usepackage{longtable}     % helps with long table options
\usepackage{mathrsfs}
% \usepackage{multicol}
\usepackage{multirow}
\usepackage{nicefrac}
% \usepackage{bm}            % special 'bold-math' package
\usepackage{scalerel}
\usepackage{siunitx}
\usepackage{subcaption}
\usepackage{tabularx}
% \usepackage{titlesec}
\usepackage{url}
\usepackage{xfrac}
\usepackage{verbatim}			% for comment environment
%\usepackage{asymptote}     % For typesetting of mathematical illustrations
%\usepackage{thumbpdf}
\usepackage[colorlinks=true]{hyperref}  % this package should be added after all others
\hypersetup{
pdftitle={Project Report Template},
linkbordercolor=orange,
citebordercolor=blue,
% pdfborder={0 0 1},
urlbordercolor=blue
}
\usepackage[textsize=footnotesize]{todonotes}
\usepackage{tikz}
\usetikzlibrary{graphs,quotes,arrows,patterns,decorations.pathmorphing, calc}

%Next line moves text up a bit without changing text area size  
\addtolength\topmargin{-.5\topmargin} %increases the top margin by half.



\makeatletter
\newcommand{\rmnum}[1]{\romannumeral #1}
\newcommand{\Rmnum}[1]{\expandafter\@slowromancap\romannumeral #1@}
\makeatother


\newcommand{\bmat}[1]{\begin{bmatrix}#1\end{bmatrix}}
\newcommand{\ubar}[1]{\text{\b{$#1$}}}
\newcommand{\norm}[2]{\|{#1}\|_{{}_{#2}}}
\newcommand{\abs}[1]{\left|{#1}\right|}

\newcommand{\mbf}[1]{\mathbf{#1}}
\newcommand{\mc}[1]{\mathcal{#1}}
\newcommand{\dd}{\operatorname{d}\!}
\newcommand{\muc}[2]{\multicolumn{#1}{c}{#2}}
\newcommand*\Eval[3]{\left.#1\right\rvert_{#2}^{#3}}
\newcommand{\inner}[1]{\left\langle#1\right\rangle}
\newcommand{\pd}[2]{\frac{\partial #1}{\partial #2}}
\newcommand{\pdd}[2]{\frac{\partial^2 #1}{\partial #2^2}}
\newcommand{\vectornorm}[1]{\left|\left|#1\right|\right|}
\newcommand\sbullet[1][.5]{\mathbin{\vcenter{\hbox{\scalebox{#1}{$\bullet$}}}}}

\newtheorem{defn}{Definition}
\newtheorem*{thm*}{Theorem}
\newtheorem{thm}{Theorem}[section]
\newtheorem{lem}[thm]{Lemma}
\newtheorem{prop}{Proposition}[section]
\newtheorem{rem}{Remark}
% \newtheorem*{lem2}{Lemma}
% \newtheorem*{prop2}{Proposition}
\newtheorem{prop3}{Proposition}

\DeclareMathOperator{\Tr}{tr}
