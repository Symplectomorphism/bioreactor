\vspace{-1em}
\section{Results}
\label{sec:results}
\vspace{-1em}

We provide extensive simulation and experimental data and their interpretation,
supporting that the proposed analog signal generator works as intended.

\vspace{-1em}
\subsection{Simulation}
\vspace{-1em}

We perform a realistic simulation with a second-order low-pass filter we came up
with in Section~\ref{ssec:second}. One of the important aspects of this design
is the selection of the op-amp. In order to keep the common-mode voltage at
$0$\unit{\volt} we choose the op-amps as CMOS type. One such op-amp is
\texttt{OP292}~\cite{op292}, which is used in this simulation.

The PWM signal generated by Teensy~\cite{teensy} is simulated exactly with a
carrier frequency of $36.6$\unit{\kilo\hertz} modulating the signal \[
V_{\text{pwm}} = \nicefrac{3.3}{2} + \nicefrac{3.3}{2}\sin{(2\pi 90 t)}.\]
Finally, the impedance of the load (PI's controller) is read off from its
datasheet and inserted as a $100$\unit{\kilo\ohm} resistance. The circuit that
is simulated using LTSpice~\cite{ltspice} is presented in
Figure~\ref{fig:real_sig_gen}.

\begin{figure}[htb] 
\includegraphics[width=8cm]{./figures/circuit.png}
\caption{The signal generator circuit in LTSpice} 
\label{fig:real_sig_gen}
\end{figure}

The simulation generates the relevant voltage responses, provided in
Figure~\ref{fig:response}. The top plot shows the PWM signal generated by Teensy
modulating a sine-wave at $90$\unit{\hertz} frequency. The individual plots in
the middle show the output of the first (cyan) and the second (purple) RC
low-pass filters ($v_f$ and $v_i$, respectively) that extract the modulated
signal from its PWM representation. Finally, the last plot shows the amplified
signal ($\text{gain}=3$) through the op-amp \texttt{OP292}. This signal is ready
to be sent to the PI controller.

\begin{figure}[t]
\includegraphics[width=0.5\textwidth]{./figures/pwm_filtered_one_two_final_signal.png}
\caption{The response from the simulation for one full period.} 
\label{fig:response}
\end{figure}


The performance of the Sallen-Key architecture from Section~\ref{ssec:sallenkey}
is shown in Figure~\ref{fig:sallenkey_sim}, where the values for the resistances
were taken to be $R_d = 68$\unit{\kilo\ohm} and capacitances to be $C_d =
1$\unit{\nano\farad}. Even though the performance of the Sallen-Key filter of
Section~\ref{ssec:sallenkey} looks very similar to the $RC$+buffer filter of
Secton~\ref{ssec:second} in simulation, in the experiments the Sallen-Key filter
outperforms significantly. We will implement this filter in our final design.

\begin{figure}[t]
\includegraphics[width=0.5\textwidth]{./figures/sallenkey.png}
\caption{The response of the Sallen-Key architecture.}
\label{fig:sallenkey_sim}
\end{figure}


\vspace{-1em}
\subsection{Experiment}
\vspace{-1em}

The design is implemented on a simple setup on my table top by lighting two red
LEDs, first one using the filtered PWM and the second using the amplified
signal, both obtained using the circuit in Section~\ref{ssec:sallenkey}.

% \begin{figure}[t]
% \includegraphics[width=0.5\textwidth]{./figures/prototype.jpg}
% \caption{Prototype working on a LED} 
% \label{fig:exp}
% \end{figure}

\begin{figure}[h]
    \includegraphics[width=0.5\textwidth]{./figures/output_osc.jpg}
    \caption{The response of the second-order Sallen-Key filter.}
    \label{fig:sallenkey_osc}
\end{figure}

Figure~\ref{fig:sallenkey_osc} shows the response of the second-order Sallen-Key
LPF observed through an oscilloscope. This is the output of the signal generator
in response to the Teensy generated $90$\unit{\hertz} $0-3.3$\unit{\volt} PWM
signal modulating the desired sine wave. This response is satisfactory and
should drive the PI controller without any issues.
