\vspace{-1em}
\section{Full Circuit Drawing of the Final Design}
\label{sec:fulldesign}
\vspace{-1em}

The final signal generator circuit schematic is presented in
Figure~\ref{fig:final_circuit}. The potential $v_t$ denotes the
$0-3.3$\unit{\volt} PWM signal modulating a $90$\unit{\hertz} sinusoidal wave at
a carrier frequency of $36.6$\unit{\kilo\hertz}. The potential $v_o$ denotes the
$0-10$\unit{\volt} analog $90$\unit{\hertz} sine-wave signal ready to be sent to
the Physik Instrumente controller E-610, depicted in the figure as the impedance
$R_{\text{load}}$, as its reference input. The capacitances and the resistances
are provided in Table~\ref{tab:expvalues}.

\begin{figure*}[b]
\begin{circuitikz}[scale=2, node distance=0.1mm and 0.1mm, rotate=-90, transform
    shape]
\draw (5,.5) node [op amp] (opamp) {\texttt{LM358}}
(opamp.down) -- ++(0,-0.25) node[ground] {}
(0,0) node [left] {$v_t$} to [R, l=$R_1$, o-*] (2,0) node[below]{$v_m$} 
to [R, l=$R_2$, *-*] (opamp.+)
to [C, l_=$C_2$, *-] ($(opamp.+)+(0,-2)$) node [ground] {}
(opamp.out) |- (3.5,2) to [C, l_=$C_1$, *-] (2,2) to [short] (2,0)
(opamp.-) -| (3.5,2)
% (opamp.out) to [short, *-*] (6.5,.5) node (vi) [above] {$v_i$};
(opamp.out) node (vi) [above right = of opamp.out]{$v_i$};
\draw[-latex] (opamp.up) -- ++(0,0.5) node [above] {$V_+$};

\draw (opamp.out) to[short, *-] ++(0.5,0) node[op amp, noinv input up, anchor=+]
(opamp2) {\texttt{LM358}}
(opamp2.-) -- ++(0, -2.0) coordinate (tmp) to [R, l=$R_{in}$, *-] ++(0,-2) node
[ground] (gnd) {} (tmp) to [R, l=$R_f$, -*] (tmp -| opamp2.out) -- (opamp2.out)
to [short, *-o] ++(1,0) node[above]{$v_o$};

\draw let 
    \p1 = (tmp), 
    \p2 = (opamp2.out)
    in
    (\x2, \y1) to [R, l=$R_{\text{load}}$] ++(0, -2) node[ground] {};

\draw (opamp2.down) -- ++(0,-0.25) node[ground] {};
\draw[-latex] (opamp2.up) -- ++(0,0.5) node [above] {$V_+$};

\end{circuitikz}
\caption{The final signal generator circuit design.}
\label{fig:final_circuit}
\end{figure*}


