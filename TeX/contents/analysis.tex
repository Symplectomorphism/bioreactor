\section{Theoretical Analysis}
\vspace{-1em}

We perform a basic PWM filter design to generate the driving $0-10$\unit{\volt}
sine wave signal to be fed into the Physik Instrumente (PI)'s controller for one
of their piezoelectric actuators~\cite{pie610}.

\vspace{-1em}
\subsection{Basic Lowpass Filter Design}
\vspace{-1em}

Consider the first-order filter sketched in Figure~\ref{fig:RC} with a
sinusoidal driving voltage. The current $i$ over the capacitor is $i =
C\frac{\dd v_o}{\dd t}$. KVL around the loop gives 
%
\begin{equation*}
    RC\frac{\dd v_o}{\dd t} + v_o = v_{\text{sig}} = A \cos{(\omega t)}
%     \label{eq:de}
\end{equation*} 
%
This differential equation has the transfer function \[ G(s) = \frac{1} {RCs+1}.
\] The steady-state solution of the differential equation is obtained as
%
\begin{align*}
    \begin{split}
    v_o(t) &= A\abs{G(j \omega)}\cos{(\omega t + \angle G(j\omega))} \\
           &= \frac{A}{\sqrt{1+\omega^2R^2C^2}}\cos{\left(\omega t -
           \arctan{(\omega R C})\right)}
    \end{split}
%    \label{eq:de_sol}
\end{align*}
%
Since we do not want our signal to be attenuated by the low-pass filter, we must
choose the values of $R$ and $C$ such that $\omega R C \ll 1$ or $2\pi RC \ll
\nicefrac{1}{f}$. 
\begin{figure}
\begin{circuitikz}[scale=0.75]
    \draw (0,0) to [vsourcesin, l=$v_{\text{sig}}$, fill=green!70!red] (0,3) to
    [R=$R$] (2,3) to [C=$C$] (2,0) -- (2,0) -- (0,0);

    \draw (2,3) to [short, *-o] ++(1,0) node[above]{$v_o$};
    % \draw (2,0) to [short, *-o] ++(1,0);
    \node [ground] at (2,0) {};
\end{circuitikz}
\caption{A first-order low-pass filter circuit.}
\label{fig:RC}
\end{figure}

Unfortunately, our actual input from the microcontroller is not a pure sine
wave, rather a PWM signal. Therefore, we also need the value of $2\pi RC$ to be
large so that it attenuates the high frequencies present in the PWM signal. This
requirement is difficult to achieve with just a first-order low-pass filter,
leading us to design a second-order low-pass filter in the next subsections.

\vspace{-1em}
\subsection{Second-Order LPF Design}
\label{ssec:second}
\vspace{-1em}

A second-order filter is implemented as a linear operator from the
Teensy-generated PWM voltage input $v_t$ to the voltage $v_i$, as presented in
Figure~\ref{fig:sig_circuit}. The remainder of this circuit constitutes a
noninverting op-amp that amplifies the sine wave extracted from its PWM
modulation from $0-3.3$\unit{\volt} to $0-10$\unit{\volt}. We analyze the
circuit so as to figure out the values of the various resistances and
capacitances.

\begin{figure}[h]
\begin{center}
\begin{circuitikz}[scale=0.535, transform shape]
    \draw (0,3) to[vsourcesin, name=vs, fill=green!70!red] (0,6);
    \node [below left, align=center, inner sep=12pt] at (vs.e)
    {$0-3.3$\unit{\volt}\\$90$\unit{\hertz} PWM};
    \node [right, align=center, inner sep=12pt] at (vs.e) {$v_t$};
    \draw (0,6) to [R=$R$] (3,6) -- ++(0.0,0) node[label={above:$v_m$}] (vm) {}
    to[short, *-] ++(0,0) to [C=$C$] (3,3) to node[ground]{} (3,3);
    \node [ground] at (0,3) {};

    \draw (3,6) to[short, *-] ++(1.0,0) node[op amp, noinv input up,
    anchor=+](A1) {\texttt{LM358}};
%     \draw (A1.+) to[short] (3,6);
    \draw[-latex] (A1.up) -- ++(0,0.5) node [above] {$V_+$};
    \draw (A1.down) -- ++(0,-0.25) node[ground] {};
    \draw (A1.-) -- ++(0,-1.25) coordinate (tmp1);
    \draw (A1.out) |- (tmp1);

    \draw (A1.out) to[short] ++(0.5,0) to [R=$R$] (8.5,5.51) -- ++(0.5,0)
    node[label={above:$v_i$}] {} to[short, *-] ++(0.0,0)
    to [C=$C$] (9,3) to [ground] (9, 3);
    \node [ground] at (9.0, 3) {};

    \draw (9,5.51) to[short, *-] ++(1.0,0) node[op amp, noinv input up,
    anchor=+](A2) {\texttt{LM358}}
    (A2.-) -- ++(0, -2.0) coordinate (tmp2) to [R, l=$R_{in}$, *-] ++(0,-2) node
    [ground] (gnd) {} (tmp2) to [R, l=$R_f$, -*] (tmp2 -| A2.out) -- (A2.out)
    to [short, *-o] ++(1,0) node[above]{$v_o$};

    \draw let 
    \p1 = (tmp2), 
    \p2 = (A2.out)
    in
    (\x2, \y1) to [R, l=$R_{\text{load}}$] ++(0, -2) node[ground] {};

    \draw (A2.down) -- ++(0,-0.25) node[ground] {};
    \draw[-latex] (A2.up) -- ++(0,0.5) node [above] {$V_+$};
\end{circuitikz}
\end{center}
\caption{The signal generator circuit.}
\label{fig:sig_circuit}
\end{figure}


The $0-3.3$\unit{\volt} PWM signal is to be filtered to extract the modulated
sine wave. Thanks to the buffer op-amp, the transfer function from the Teensy
input $v_t$ to the input $v_i$ to the non-inverting amplifier op-amp is given by
%
\begin{equation}
H(s) = \frac{V_i(s)}{V_t(s)} = \frac{1}{R^2C^2s^2+2RCs+1}.
\label{eq:tf}
\end{equation}
%
This is a fully-damped transfer function with poles at $s_{1,2} =
-\nicefrac{1}{RC}$. In other words, the cut-off frequency of this filter is at
$f = \frac{1}{2\pi RC}$ with a roll-off of $40$\unit{\decibel} per decade.
Contrast this to the first-order filter of the previous subsection where the
roll-off was $20$\unit{\decibel} per decade. The greater roll-off rate allows us
be able to select $2\pi RC \ll \nicefrac{1}{f}$ while simultaneously achieving
excellent high-frequency attenuation.

Our desired signal is a $90$\unit{\hertz} sinusoidal. We do not want to lose
this signal, i.e., we want our transfer function $H(s)$ to approximately have a
unity gain at this frequency. We set $2\pi RC \leq \nicefrac{1}{900}$ or $RC
\leq 1.768 \times 10^{-4}$\unit{\ohm\farad} to satisfy this requirement. Using
some standard values of the resistance $R = 100$\unit{\kilo\ohm} and the
capacitance $C = 1$\unit{\nano\farad}, we obtain $RC = 1.0 \times
10^{-4}$\unit{\ohm\farad}, meeting the specification. The attenuation at
Teensy's PWM carrier frequency of $36.6$\unit{\kilo\hertz} is found by \[ 20
\log_{10}\left\{\abs{H(j 2\pi 36600)}\right\} = -54.483\,\unit{\decibel}. \]

Lastly, we want to amplify the input voltage $v_i$ thrice in order to hit the
$0-10$\unit{\volt} mark. The gain of the non-inverting amplifier is $k = 1 +
\nicefrac{R_f}{R_{in}}$. We choose $R_f = 2$\unit{\kilo\ohm} and $R_{in} =
1$\unit{\kilo\ohm} to achieve $k = 3$. The high gain bandwidth product of the
\texttt{LM358} op-amp is read from its datasheet to be $\text{GBP} =
1.2$\unit{\mega\hertz}. Hence the transfer function from the input voltage $v_i$
to the output voltage $v_o$ that will be applied to the PI controller is
approximately given by \[ \frac{V_o(s)}{V_i(s)} =
\frac{k}{\frac{k}{2\pi\text{GBP}}s + 1} \approx \frac{3}{3.979\times 10^{-7}s +
1}, \] which will have a firm unity gain at our desired oscillation frequency of
$90$\unit{\hertz}.


% \vspace{-1em}
% \subsection{Adding a HPF for DC removal}
% \vspace{-1em}
% 
% Any remaining DC component of the signal at $v_i$ may be removed by adding a
% capacitor between the inverting terminal of the amplifier op-amp and the ground
% as depicted in Figure~\ref{fig:hpf}. Assuming an ideal op-amp for the moment,
% the transfer function between $v_o$ and $v_i$ is given by
% \[\frac{V_o(s)}{V_i(s)} = \frac{R_fC_{in}s}
% 
% 
% % \begin{figure}[h]
% % \begin{circuitikz}[]
% %     \draw (0,1) node[label={above:$v_i$}] {} to[short, o-] ++(0.5, 0) to
% %     [C=$C_1$] ++(1.0,0) -| ++(0.5, 0) -- ++(0.0,0)
% %     node[label={above:$\tilde{v}_i$}] {} to[short, *-] ++(0,-0.25) to [R=$R_1$]
% %     (2,-1.0) to node[ground]{} (2,-1.0);
% %     
% %     \node [op amp, noinv input up](A2) at (4.5, 0.51) {\texttt{OP292}};
% %     \draw (A2.+) to[short] (2.0,1.0);
% %     \draw[-latex] (A2.up) -- ++(0,0.5) node [above] {$V_+$};
% %     \draw (A2.down) -- ++(0,-0.25) node[ground] {};
% %     \draw (A2.-) -- ++(0,-2.0) coordinate(FB) to [R=$R_{in}$] (3.3, -4)
% %     coordinate (tmp) node[ground]{} (FB) to [R=$R_f$, *-] (FB -| A2.out) -- 
% %     (A2.out);
% % 
% %     \draw ($ (FB) + (2.38, 0) $) to [R=$R_{\text{load}}$] ($ (tmp) + (2.38, 0)
% %     $) node[ground]{};
% % 
% %     \draw ($ (FB) + (2.38, 0) $) to[short, *-o] ++(1,0) node[above]{$v_o$};
% % \end{circuitikz}
% % \caption{High-pass filter to remove leaking DC.}
% % \label{fig:hpf}
% % \end{figure}
% 
% 
% \begin{figure}[h]
% \begin{circuitikz}[]
%     \draw (0,1) node[label={above:$v_i$}] {} to[short, o-] ++(0.5,0);
%     
%     \node [op amp, noinv input up](A2) at (2.0, 0.51) {\texttt{OP292}};
%     \draw (A2.+) to[short] (0.5,1.0);
%     \draw[-latex] (A2.up) -- ++(0,0.5) node [above] {$V_+$};
%     \draw (A2.down) -- ++(0,-0.25) node[ground] {};
%     \draw (A2.-) -- ++(0,-2.0) coordinate(FB) to [R=$R_{in}$] ++(0, -2.0) to
%     [C=$C_{in}$] (0.82, -5) coordinate (tmp) node[ground]{} (FB) to [R=$R_f$, *-]
%     (FB -| A2.out) -- (A2.out);
% 
%     \draw ($ (FB) + (2.38, 0) $) to [R=$R_{\text{load}}$] ($ (tmp) + (2.42, 0)
%     $) node[ground]{};
% 
%     \draw ($ (FB) + (2.38, 0) $) to[short, *-o] ++(1,0) node[above]{$v_o$};
% \end{circuitikz}
% \caption{High-pass filter to remove leaking DC.}
% \label{fig:hpf}
% \end{figure}

\vspace{-1em}
\subsection{Sallen-Key Architecture}
\label{ssec:sallenkey}
\vspace{-1em}

Another well-known architecture that works well for this sort of problem is the
Sallen-Key low-pass filter, which \ul{replaces} the two $RC$+buffer combination
whose output enters the amplifier op-amp, as shown in
Figure~\ref{fig:sallenkey}. The blown-up full schematic of signal generator
circuit may be found in Figure~\ref{fig:final_circuit} of the appendix.

\begin{figure}[h]
\begin{circuitikz}[scale=1]\draw
(5,.5) node [op amp] (opamp) {\texttt{LM358}}
(0,0) node [left] {$v_t$} to [R, l=$R_{d1}$, o-*] (2,0) node[below]{$v_m$} 
to [R, l=$R_{d2}$, *-*] (opamp.+)
to [C, l_=$C_{d2}$, *-] ($(opamp.+)+(0,-2)$) node [ground] {}
(opamp.out) |- (3.5,2) to [C, l_=$C_{d1}$, *-] (2,2) to [short] (2,0)
(opamp.-) -| (3.5,2)
(opamp.out) to [short, *-o] (7,.5) node [right] {$v_i$};
\end{circuitikz}
\caption{Sallen Key (second-order) low-pass filter.}
\label{fig:sallenkey}
\end{figure}

We find the governing equations of this circuit. Assume an ideal op-amp model so
that both of the inputs of the op-amp have potential $v_i$. Let the current $i$
flowing over $R_{d1}$ split into $i_1$ and $i_2$, the former flowing into
$C_{d1}$ and the latter into $C_{d2}$ through $R_{d2}$. KCL gives
%
\vspace{-1em}
\begin{align*}
    i &= i_1 + i_2 = \frac{1}{R_{d1}}(v_t - v_m), \\
    i_1 &= C_1 \frac{\dd\, (v_m - v_i)}{\dd t}, \\
    i_2 &= C_2 \frac{\dd v_i}{\dd t}.
\end{align*}
%
KVL around the bottom loop ($v_m - v_o - \text{gnd}$) gives $v_m = R_{d2}i_2 +
v_i = R_{d2}C_{d2}\frac{\dd v_i}{\dd t} + v_i$. Plugging this into the second
equation gives $i_1 = R_{d2}C_{d1}C_{d2}\frac{\dd^2v_i}{\dd t^2}$. Combined with
the first and third equations, this yields
%
\begin{equation}
    R_{d1}R_{d2}C_{d1}C_{d2}\ddot{v}_i + (R_{d1} + R_{d2})C_{d2}\dot{v}_i + v_i
    = v_t.
    \label{eq:sallenkey_de}
\end{equation}
%
We will take $R_{d1} = R_{d2} = R_d$ and $C_{d1} = C_{d2} = C_d$ so that the
transfer function for this linear differential equation is observed to coincide
with equation~\eqref{eq:tf} of Section~\ref{ssec:second}: \[ H(s) =
\frac{V_i(s)}{V_t(s)} = \frac{1}{R_d^2C_d^2s^2 + 2R_dC_ds + 1}.\]
%
Therefore, all the analysis that follows equation~\eqref{eq:tf} goes through for
the Sallen-Key architecture as well. Theory, simulation and experiments show
that some good values for the resistances $R_d$ and the capacitances $C_d$ are
$R_d = 100$\unit{\kilo\ohm} and $C_d = 1$\unit{\nano\farad}, with a cutoff
frequency of $f \approx 1592$\unit{\hertz}. There is a range of resistance and
capacitance values that work well around these nominal values.

\begin{rem}
    This is the filter to be implemented in the final design.
\end{rem}
