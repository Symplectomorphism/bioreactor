\section{Theoretical Analysis}
\vspace{-1em}

We perform a basic PWM filter design to generate the driving $0-10$\unit{\volt}
sine wave signal to be fed into the Physik Instrumente (PI)'s controller for one
of their piezoelectric actuators~\cite{pie610}.

\vspace{-1em}
\subsection{Basic Lowpass Filter Design}
\vspace{-1em}

Consider the first-order filter sketched in Figure~\ref{fig:RC} with a
sinusoidal driving voltage. The current $i$ over the capacitor is $i =
C\frac{\dd v_o}{\dd t}$. KVL around the loop gives 
%
\begin{equation*}
    RC\frac{\dd v_o}{\dd t} + v_o = v_{\text{sig}} = A \cos{(\omega t)}
%     \label{eq:de}
\end{equation*} 
%
This differential equation has the transfer function \[ G(s) = \frac{1} {RCs+1}.
\] The steady-state solution of the differential equation is obtained as
%
\begin{align*}
    \begin{split}
    v_o(t) &= A\abs{G(j \omega)}\cos{(\omega t + \angle G(j\omega))} \\
           &= \frac{A}{\sqrt{1+\omega^2R^2C^2}}\cos{\left(\omega t -
           \arctan{(\omega R C})\right)}
    \end{split}
%    \label{eq:de_sol}
\end{align*}
%
Since we do not want our signal to be attenuated by the low-pass filter, we must
choose the values of $R$ and $C$ such that $\omega R C \ll 1$ or $2\pi RC \ll
\nicefrac{1}{f}$. 
\begin{figure}
\begin{circuitikz}[scale=0.75]
    \draw (0,0) to [vsourcesin, l=$v_{\text{sig}}$, fill=green!70!red] (0,3) to
    [R=$R$] (2,3) to [C=$C$] (2,0) -- (2,0) -- (0,0);

    \draw (2,3) to [short, *-o] ++(1,0) node[above]{$v_o$};
    % \draw (2,0) to [short, *-o] ++(1,0);
    \node [ground] at (2,0) {};
\end{circuitikz}
\caption{A first-order low-pass filter circuit.}
\label{fig:RC}
\end{figure}

Unfortunately, our actual input from the microcontroller is not a pure sine
wave, rather a PWM signal. Therefore, we also need the value of $2\pi RC$ to be
large so that it attenuates the high frequencies present in the PWM signal. This
requirement is difficult to achieve with just a first-order low-pass filter,
leading us to design a second-order low-pass filter in the next subsections.

\vspace{-1em}
\subsection{Second-Order LPF Design}
\label{ssec:second}
\vspace{-1em}

A second-order filter is implemented as a linear operator from the
Teensy-generated PWM voltage input $v_t$ to the voltage $v_i$, as presented in
Figure~\ref{fig:sig_circuit}. The remainder of this circuit constitutes a
noninverting op-amp that amplifies the sine wave extracted from its PWM
modulation from $0-3.3$\unit{\volt} to $0-10$\unit{\volt}. We analyze the
circuit so as to figure out the values of the various resistances and
capacitances.

\begin{figure}[h]
\begin{center}
\begin{circuitikz}[scale=0.535, transform shape]
    \draw (0,3) to[vsourcesin, name=vs, fill=green!70!red] (0,6);
    \node [below left, align=center, inner sep=12pt] at (vs.e)
    {$0-3.3$\unit{\volt}\\$90$\unit{\hertz} PWM};
    \node [right, align=center, inner sep=12pt] at (vs.e) {$v_t$};
    \draw (0,6) to [R=$R$] (3,6) -- ++(0.0,0) node[label={above:$v_m$}] (vm) {}
    to[short, *-] ++(0,0) to [C=$C$] (3,3) to node[ground]{} (3,3);
    \node [ground] at (0,3) {};

    \draw (3,6) to[short, *-] ++(1.0,0) node[op amp, noinv input up,
    anchor=+](A1) {\texttt{LM358}};
%     \draw (A1.+) to[short] (3,6);
    \draw[-latex] (A1.up) -- ++(0,0.5) node [above] {$V_+$};
    \draw (A1.down) -- ++(0,-0.25) node[ground] {};
    \draw (A1.-) -- ++(0,-1.25) coordinate (tmp1);
    \draw (A1.out) |- (tmp1);

    \draw (A1.out) to[short] ++(0.5,0) to [R=$R$] (8.5,5.51) -- ++(0.5,0)
    node[label={above:$v_i$}] {} to[short, *-] ++(0.0,0)
    to [C=$C$] (9,3) to [ground] (9, 3);
    \node [ground] at (9.0, 3) {};

    \draw (9,5.51) to[short, *-] ++(1.0,0) node[op amp, noinv input up,
    anchor=+](A2) {\texttt{LM358}}
    (A2.-) -- ++(0, -2.0) coordinate (tmp2) to [R, l=$R_{in}$, *-] ++(0,-2) node
    [ground] (gnd) {} (tmp2) to [R, l=$R_f$, -*] (tmp2 -| A2.out) -- (A2.out)
    to [short, *-o] ++(1,0) node[above]{$v_o$};

    \draw let 
    \p1 = (tmp2), 
    \p2 = (A2.out)
    in
    (\x2, \y1) to [R, l=$R_{\text{load}}$] ++(0, -2) node[ground] {};

    \draw (A2.down) -- ++(0,-0.25) node[ground] {};
    \draw[-latex] (A2.up) -- ++(0,0.5) node [above] {$V_+$};
\end{circuitikz}
\end{center}
\caption{The signal generator circuit.}
\label{fig:sig_circuit}
\end{figure}


The $0-3.3$\unit{\volt} PWM signal is to be filtered to extract the modulated
sine wave. Thanks to the buffer op-amp, the transfer function from the Teensy
input $v_t$ to the input $v_i$ to the non-inverting amplifier op-amp is given by
%
\begin{equation*}
H(s) = \frac{V_i(s)}{V_t(s)} = \frac{1}{R^2C^2s^2+2RCs+1}.
% \label{eq:tf}
\end{equation*}
%
This is a critically-damped transfer function with both poles at $s_{1,2} =
-\nicefrac{1}{RC}$. In other words, the cut-off frequency of this filter is at
$f_n = \frac{1}{2\pi RC}$ with a roll-off of $40$\unit{\decibel} per decade.
Contrast this to the first-order filter of the previous subsection where the
roll-off was $20$\unit{\decibel} per decade. The greater roll-off rate allows us
be able to select $2\pi RC \ll \nicefrac{1}{f}$ while simultaneously achieving
excellent high-frequency attenuation.

Our desired signal is a $90$\unit{\hertz} sinusoidal, which should not be
attenuated by the low-pass filter, i.e., the transfer function $H(s)$ should
have approximately a unity gain at this frequency. To that end, we set $f_n = 5
\times 90 = 450$\unit{\hertz}, which produces the constraint $RC \leq
\nicefrac{1}{2\pi f_n}$.

The attenuation needed at the PWM carrier frequency $f_c =
36.6$\unit{\kilo\hertz} provides us with a lower bound on the product $RC$. 
%\[ 20 \log{\abs{H(j\omega)}} = -20 \log{\left(1 + 
% R^2C^2\omega^2\right)}\,\unit{\decibel}. \] 
If we want no less than $\alpha$-fold attenuation, then we must have that 
%
\begin{align*}
    -20 \log \alpha &\geq 20 \log{\abs{H(j2\pi f_c)}} = -20 \log{(1+4\pi^2
    R^2C^2f_c^2)} \\ &\Rightarrow 1 + 4 \pi^2 R^2C^2f_c^2 \geq \alpha \Rightarrow
    RC \geq \frac{\sqrt{\alpha-1}}{2\pi f_c}.
\end{align*}
%
We summarize the lower and upper bounds on $RC$:
%
\begin{equation*}
    \frac{\sqrt{\alpha-1}}{2\pi f_c} \leq RC \leq \frac{1}{2\pi f_n}.
\end{equation*}

% Using
% some standard values of the resistance $R = 100$\unit{\kilo\ohm} and the
% capacitance $C = 1$\unit{\nano\farad}, we obtain $RC = 1.0 \times
% 10^{-4}$\unit{\second}, meeting the specification. The attenuation at
% Teensy's PWM carrier frequency of $36.6$\unit{\kilo\hertz} is found by \[ 20
% \log_{10}\left\{\abs{H(j 2\pi 36600)}\right\} = -54.483\,\unit{\decibel}. \]

Lastly, we want to amplify the input voltage $v_i$ thrice in order to hit the
$0-10$\unit{\volt} mark. The gain of the non-inverting amplifier is $k = 1 +
\nicefrac{R_f}{R_{in}}$. We choose $R_f, R_{in} \geq 1$\unit{\kilo\ohm} so as to
achieve $k = 3$. The high gain bandwidth product of the \texttt{LM358} op-amp is
read from its datasheet to be $\text{GBP} = 1.2$\unit{\mega\hertz}. Hence the
transfer function from the input voltage $v_i$ to the output voltage $v_o$ that
will be applied to the PI controller is approximately given by \[
    \frac{V_o(s)}{V_i(s)} = \frac{k}{\frac{k}{2\pi\text{GBP}}s + 1} \approx
\frac{3}{3.979\times 10^{-7}s + 1}, \] which will have a firm unity gain at our
desired oscillation frequency of $90$\unit{\hertz}.


% \vspace{-1em}
% \subsection{Adding a HPF for DC removal}
% \vspace{-1em}
% 
% Any remaining DC component of the signal at $v_i$ may be removed by adding a
% capacitor between the inverting terminal of the amplifier op-amp and the ground
% as depicted in Figure~\ref{fig:hpf}. Assuming an ideal op-amp for the moment,
% the transfer function between $v_o$ and $v_i$ is given by
% \[\frac{V_o(s)}{V_i(s)} = \frac{R_fC_{in}s}
% 
% 
% % \begin{figure}[h]
% % \begin{circuitikz}[]
% %     \draw (0,1) node[label={above:$v_i$}] {} to[short, o-] ++(0.5, 0) to
% %     [C=$C_1$] ++(1.0,0) -| ++(0.5, 0) -- ++(0.0,0)
% %     node[label={above:$\tilde{v}_i$}] {} to[short, *-] ++(0,-0.25) to [R=$R_1$]
% %     (2,-1.0) to node[ground]{} (2,-1.0);
% %     
% %     \node [op amp, noinv input up](A2) at (4.5, 0.51) {\texttt{OP292}};
% %     \draw (A2.+) to[short] (2.0,1.0);
% %     \draw[-latex] (A2.up) -- ++(0,0.5) node [above] {$V_+$};
% %     \draw (A2.down) -- ++(0,-0.25) node[ground] {};
% %     \draw (A2.-) -- ++(0,-2.0) coordinate(FB) to [R=$R_{in}$] (3.3, -4)
% %     coordinate (tmp) node[ground]{} (FB) to [R=$R_f$, *-] (FB -| A2.out) -- 
% %     (A2.out);
% % 
% %     \draw ($ (FB) + (2.38, 0) $) to [R=$R_{\text{load}}$] ($ (tmp) + (2.38, 0)
% %     $) node[ground]{};
% % 
% %     \draw ($ (FB) + (2.38, 0) $) to[short, *-o] ++(1,0) node[above]{$v_o$};
% % \end{circuitikz}
% % \caption{High-pass filter to remove leaking DC.}
% % \label{fig:hpf}
% % \end{figure}
% 
% 
% \begin{figure}[h]
% \begin{circuitikz}[]
%     \draw (0,1) node[label={above:$v_i$}] {} to[short, o-] ++(0.5,0);
%     
%     \node [op amp, noinv input up](A2) at (2.0, 0.51) {\texttt{OP292}};
%     \draw (A2.+) to[short] (0.5,1.0);
%     \draw[-latex] (A2.up) -- ++(0,0.5) node [above] {$V_+$};
%     \draw (A2.down) -- ++(0,-0.25) node[ground] {};
%     \draw (A2.-) -- ++(0,-2.0) coordinate(FB) to [R=$R_{in}$] ++(0, -2.0) to
%     [C=$C_{in}$] (0.82, -5) coordinate (tmp) node[ground]{} (FB) to [R=$R_f$, *-]
%     (FB -| A2.out) -- (A2.out);
% 
%     \draw ($ (FB) + (2.38, 0) $) to [R=$R_{\text{load}}$] ($ (tmp) + (2.42, 0)
%     $) node[ground]{};
% 
%     \draw ($ (FB) + (2.38, 0) $) to[short, *-o] ++(1,0) node[above]{$v_o$};
% \end{circuitikz}
% \caption{High-pass filter to remove leaking DC.}
% \label{fig:hpf}
% \end{figure}

\vspace{-1em}
\subsection{Sallen-Key Architecture}
\label{ssec:sallenkey}
\vspace{-1em}

Another well-known architecture that works well for this sort of problem is the
Sallen-Key low-pass filter, which \ul{replaces} the two $RC$+buffer combination
whose output enters the amplifier op-amp, as shown in
Figure~\ref{fig:sallenkey}. The blown-up full schematic of signal generator
circuit may be found in Figure~\ref{fig:final_circuit} of the appendix.

\begin{figure}[h]
\begin{circuitikz}[scale=1]\draw
(5,.5) node [op amp] (opamp) {\texttt{LM358}}
(0,0) node [left] {$v_t$} to [R, l=$R_{1}$, o-*] (2,0) node[below]{$v_m$} 
to [R, l=$R_{2}$, *-*] (opamp.+)
to [C, l_=$C_{2}$, *-] ($(opamp.+)+(0,-2)$) node [ground] {}
(opamp.out) |- (3.5,2) to [C, l_=$C_{1}$, *-] (2,2) to [short] (2,0)
(opamp.-) -| (3.5,2)
(opamp.out) to [short, *-o] (7,.5) node [right] {$v_i$};
\end{circuitikz}
\caption{Sallen Key (second-order) low-pass filter.}
\label{fig:sallenkey}
\end{figure}

We find the governing equations of this circuit. Assume an ideal op-amp model so
that both of the inputs of the op-amp have potential $v_i$. Let the current $i$
flowing over $R_{1}$ split into $i_1$ and $i_2$, the former flowing into
$C_{1}$ and the latter into $C_2$ through $R_2$. KCL gives
%
% \vspace{-1em}
\begin{align*}
    i &= i_1 + i_2 = \frac{1}{R_{1}}(v_t - v_m), \\
    i_1 &= C_{1} \frac{\dd\, (v_m - v_i)}{\dd t}, \\
    i_2 &= C_2 \frac{\dd v_i}{\dd t}.
\end{align*}
%
KVL around the bottom loop ($v_m - v_i - \text{gnd}$) gives $v_m = R_2i_2 +
v_i = R_2C_2\frac{\dd v_i}{\dd t} + v_i$. Plugging this into the second
equation gives $i_1 = R_2C_{1}C_2\frac{\dd^2v_i}{\dd t^2}$. Combined with
the first and third equations, this yields
%
\begin{equation}
    R_{1}R_2C_{1}C_2\ddot{v}_i + (R_{1} + R_2)C_2\dot{v}_i + v_i
    = v_t.
    \label{eq:sallenkey_de}
\end{equation}
%
The transfer function of the Sallen-Key filter is read as
%
\begin{equation}
    T(s) = \frac{1}{R_{1}R_2C_{1}C_2s^2 + (R_{1}+R_2)C_2s + 1}.
    \label{eq:tf_sallenkey}
\end{equation}
%
The gain and phase of this transfer function may be computed as follows
%
\begin{align}
    \begin{split}
    \abs{T(j\omega)} &= \frac{1}{\sqrt{1+R_1^2R_2^2C_1^2C_2^2\omega^4}}, \\
    \angle{T(j\omega)} &=
    -\arctan{\frac{(R_1+R_2)C_2\omega}{1-R_1R_2C_1C_2\omega^2}}.
    \end{split}
    \label{eq:sallenkey_gainphase}
\end{align}

We want to make this a Butterworth filter, making its frequency response as flat
as possible in the passband. To that end, we identify the characteristic
polynomial as
\[s^2 + \frac{R_1+R_2}{R_1R_2C_2}s + \frac{1}{R_1R_2C_1C_2} = s^2 +
2\zeta\omega_ns + \omega_n^2,\] with the damping coefficient $\zeta =
\nicefrac{1}{\sqrt{2}}$. This requirement on the damping coefficient translates
to the constraint on the circuit elements:
%
\begin{equation}
    \frac{(R_1+R_2)^2}{2R_1R_2} = \frac{C_1}{C_2}. 
    \label{eq:cons1}
\end{equation}
%
This Butterworth filter requirement simplifies the gain and
phase~\eqref{eq:sallenkey_gainphase} of the transfer
function~\eqref{eq:tf_sallenkey} as
%
\begin{align}
    \begin{split}
    \abs{T(j\omega)} &= \frac{1}{\sqrt{1+\nicefrac{1}{4}(R_1+R_2)^4C_2^4\omega^4}}, \\
    \angle{T(j\omega)} &=
    -\arctan{\frac{2(R_1+R_2)C_2\omega}{2-(R_1+R_2)^2C_2^2\omega^2}}.
    \end{split}
    \label{eq:gp_simplified}
\end{align}
%
We ask that the magnitude of the output signal to be very close to the input
signal, i.e., $\nicefrac{1}{\alpha_1}\abs{v_t(t)} \leq \abs{v_i(t)} \leq
\abs{v_t(t)}$ at the frequency of operation $f_o$ (e.g. $90$\unit{\hertz}),
where $\alpha_1 > 1$ and $\alpha_1 \approx 1$ (e.g. $\alpha_1 = 1 +
\nicefrac{1}{1000}$). \[ \sqrt{1+\nicefrac{1}{4}(R_1+R_2)^4C_2^4(2\pi f_o)^4}
\leq \alpha_1, \] or solving for the circuit elements
%
\begin{equation}
    (R_1+R_2)C_2 \leq \frac{\sqrt[4]{\alpha_1^2-1}}{\sqrt{2}\pi f_o}.
    \label{eq:cons2}
\end{equation}
%
Asking for no less than $\alpha_2$-fold attenuation ($\alpha_2 \gg 1$) at the
PWM carrier frequency $f_c$ provides the constraint \[
\sqrt{1+\nicefrac{1}{4}(R_1+R_2)^4C_2^4(2\pi f_c)^4} \geq \alpha_2, \] or,
solving for the circuit elements,
%
\begin{equation}
    (R_1+R_2)C_2 \geq \frac{\sqrt[4]{\alpha_2^2-1}}{\sqrt{2}\pi f_c}.
    \label{eq:cons3}
\end{equation}
%
Combining the gain requirements~\eqref{eq:cons2} and~\eqref{eq:cons3}, we obtain
the design interval  
\begin{equation}
\frac{\sqrt[4]{\alpha_2^2 - 1}}{\sqrt{2}\pi f_c} \leq
(R_1+R_2)C_2 \leq \frac{\sqrt[4]{\alpha_1^2-1}}{\sqrt{2}\pi f_o}.
\label{eq:interval}
\end{equation}

To find specific values for the resistances and capacitances, we specify the
following quantities:
%
\begin{enumerate}
    \item Tolerated attenuation at operation frequency: $\alpha_1$,
    \item The value of the decoupling capacitance: $C_2$,
    \item The ratio between the resistances: $r = \nicefrac{R_1}{R_2}$.
\end{enumerate}
%
Once these are specified, the values of the remaining
circuit components can be pinpointed. Indeed, from the
constraint~\eqref{eq:cons1}, we obtain that \[C_1 = \frac{(1+r)^2}{2r}C_2. \] 
%
We make sure that the signal amplitude is attenuated no more than
a factor of $\alpha_1$ at the operating frequency as well as provides ample
attenuation at the PWM carrier frequency by satisfying the second inequality
in~\eqref{eq:interval} as equality. This gives \[ R_2 =
\frac{\sqrt[4]{\alpha_1^2-1}}{(1+r)\sqrt{2}\pi C_2} \;\; \text{ and } \;\; R_1 =
rR_2.\]

\begin{rem}
    Choosing the values for the circuit elements in this manner also determines
    the natural frequency and hence the phase shift. From the Sallen-Key filter
    transfer function~\ref{eq:tf_sallenkey}, the natural frequency is computed
    from \[f_n = \frac{1}{2\pi \sqrt{R_1R_2C_1C_2}} \Rightarrow \frac{f_o}{f_n}
    = \sqrt[4]{\alpha_1^2 -1}. \]
    %
    Further, using equation~\eqref{eq:gp_simplified}, the phase shift is
    given by \[ \angle T(j2\pi f) =
    -\arctan{\frac{\sqrt{2}\nicefrac{f}{f_n}}{1-\nicefrac{f^2}{f_n^2}}}. \]
\end{rem}

We have used this procedure to compute the values for the circuit components,
selecting $\alpha_1 = 1+\nicefrac{1}{1000}$, $C_2 = 1$\unit{\nano\farad} and $r
= 5$. The values for the remaining components are provided in
Table~\ref{tab:theoretical_values}. This is a Butterworth filter that is
maximally flat in its passband with $f_n = 425.5$\unit{\hertz} and provides
$-77.4$\unit{\decibel} attenuation at the carrier frequency. At the operation
frequency $f_o = 90$\unit{\hertz}, this filter introduces a phase shift of
$-17.39^\circ$.

{\renewcommand{\arraystretch}{1.5}
\begin{table}[t]
    \centering
    \caption{Theoretical values of resistances and capacitances.}
    \begin{tabular}{*6c}
        \toprule
        $R_1$ & $R_2$ & $C_1$ & $C_2$ & $R_f$ & $R_{in}$ \\    
        \hline
        \midrule
        $440.8$\unit{\kilo\ohm} & $88.2$\unit{\kilo\ohm} &
        $3.6$\unit{\nano\farad} & $1$\unit{\nano\farad} &
        $2$\unit{\kilo\ohm} & $1$\unit{\kilo\ohm} \\
        \bottomrule
    \end{tabular}
    \label{tab:theoretical_values}
    \vspace{-1em}
\end{table}
}

\begin{rem}
    This is the filter to be implemented in the final design. The experimental
    values of $R_{1}$, $R_2$, $C_{1}$ and $C_2$ are selected to be standard
    values that are closest to the values provided in
    Table~\ref{tab:theoretical_values}.
\end{rem}
