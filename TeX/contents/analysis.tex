\section{Theoretical Analysis}
\vspace{-1em}

We perform basic PWM filter design to generate the driving $0-10$\unit{\volt}
sine wave for the Physik Instrumente (PI)'s piezoelectric
actuator~\cite{pie610}.

\vspace{-1em}
\subsection{Basic Lowpass Filter Design}
\vspace{-1em}

Consider the first-order filter sketched in Figure~\ref{fig:RC} with a
sinusoidal driving voltage. The current $i$ over the capacitor is $i =
C\frac{\dd v_c}{\dd t}$. KVL around the circuit gives 
%
\begin{equation}
    RC\frac{\dd v_c}{\dd t} + v_c = v_{\text{sig}} = A \cos{(\omega t)}
    \label{eq:de}
\end{equation} 
%
This differential equation has the transfer function \[ G(s) = \frac{1}
{RCs+1}. \] The steady-state solution of the differential equation~\eqref{eq:de}
is obtained as
%
\begin{align}
    \begin{split}
    v_c(t) &= A\abs{G(j \omega)}\cos{(\omega t + \angle G(j\omega))} \\
           &= \frac{A}{\sqrt{1+\omega^2R^2C^2}}\cos{\left(\omega t -
           \arctan{(\omega R C})\right)}
    \end{split}
    \label{eq:de_sol}
\end{align}
%
Since we do not want our signal to be attenuated by the low-pass filter, we must
choose the values of $R$ and $C$ such that $\omega R C \ll 1$ or $2\pi RC \ll
\nicefrac{1}{f}$. 
\begin{figure}
\begin{circuitikz}[]
    \draw (0,0) to [vsource, l=$V_{\text{sig}}$] (0,3) to [R=$R$] (2,3) to
    [C=$C$] (2,0) -- (2,0) -- (0,0);

    \draw (2,3) to [short, *-o] ++(1,0) node[above]{$v_c$};
    \draw (2,0) to [short, *-o] ++(1,0);
    \node [ground] at (2,0) {};
\end{circuitikz}
\caption{An first-order low-pass filter circuit.}
\label{fig:RC}
\end{figure}

Unfortunately, our actual input from the microcontroller is not a pure sine
wave, rather a PWM signal. Therefore, we also need the value of $2\pi RC$ to be
large so that it attenuates the high frequencies present in the PWM signal. This
is difficult to achieve with just a single low-pass filter. Hence, we design a
second-order low-pass filter in the next subsection.

\vspace{-1em}
\subsection{Second-Order LPF Design}
\label{ssec:second}
\vspace{-1em}

The second-order filter is implemented in the first part of the circuit
presented in Figure~\ref{fig:sig_circuit}. The second part of this circuit
amplifies the sine wave extracted from its PWM modulation from
$0-3.3$\unit{\volt} to $0-10$\unit{\volt}. We analyze the circuit so as to
figure out the values of the various resistances and capacitances.

\begin{figure*}[t]
\begin{circuitikz}[]
    \draw (0,0) to[vsourcesin, name=vs] (0,6);
    \node [below left, align=center, inner sep=12pt] at (vs.e)
    {$0-3.3$\unit{\volt}\\$90$\unit{\hertz} PWM};
    \node [right, align=center, inner sep=12pt] at (vs.e) {$v_t$};
    \draw (0,6) to [R=$R$] (3,6) -- ++(0.0,0) node[label={above:$v_f$}] {}
    to[short, *-] ++(0,0) to [C=$C$] (3,0) to node[ground]{} (3,0);
    \node [ground] at (0,0) {};

    \node [op amp, noinv input up](A1) at (5.5, 5.51) {\texttt{OP292}};
    \draw (A1.+) to[short] (3,6);
    \draw[-latex] (A1.up) -- ++(0,0.5) node [above] {$V_+$};
    \draw (A1.down) -- ++(0,-0.25) node[ground] {};
    \draw (A1.out) to[short, *-] ++(0, -2.0) -- ++(-2.39, 0) to (A1.-);

    \draw (A1.out) to[short] ++(0.5,0) to [R=$R$] (8.5,5.51) -- ++(0.5,0)
    node[label={above:$v_i$}] {} to[short, *-] ++(0.0,0)
    to [C=$C$] (9,0) to [ground] (9, 0);
    \node [ground] at (9.0, 0) {};

    \node [op amp, noinv input up](A2) at (11.5, 5.02) {\texttt{OP292}};
    \draw (A2.+) to[short] (9,5.51);
    \draw[-latex] (A2.up) -- ++(0,0.5) node [above] {$V_+$};
    \draw (A2.down) -- ++(0,-0.25) node[ground] {};
    \draw (A2.-) -- ++(0,-2.0) coordinate(FB) to [R=$R_{in}$] (10.35, 0)
    coordinate (tmp) node[ground]{} (FB) to [R=$R_f$, *-] (FB -| A2.out) -- 
    (A2.out);

    \draw ($ (FB) + (2.38, 0) $) to [R=$R_{\text{load}}$] ($ (tmp) + (2.38, 0)
    $) node[ground]{};

    \draw ($ (FB) + (2.38, 0) $) to[short, *-o] ++(1,0) node[above]{$v_o$};
\end{circuitikz}
\caption{The signal generator circuit.}
\label{fig:sig_circuit}
\end{figure*}

We start by analyzing how the input voltage, $v_i$, to the non-inverting
amplifier responds to Teensy-generated PWM voltage, $v_t$. This is a
$0-3.3$\unit{\volt} PWM signal to be filtered to extract the modulated sine
wave. Thanks to the buffer op-amp, the transfer function from the Teensy input
$v_t$ to the input $v_i$ to the non-inverting amplifier op-amp is given by \[
\frac{V_i(s)}{V_t(s)} = H(s) = \frac{1}{R^2C^2s^2+2RCs+1}.\]
%
This is a fully-damped transfer function with poles at $s_{1,2} =
\nicefrac{1}{RC}$. In other words, the cut-off frequency of this filter is at $f
= \frac{1}{2\pi RC}$ with a roll-off of $40$\unit{\decibel} per decade. Contrast
this to the first-order filter of the previous subsection where the roll-off was
$20$\unit{\decibel} per decade. The greater roll-off rate allows us be able to
select $2\pi RC \ll \nicefrac{1}{f}$ while simultaneously achieving excellent
high-frequency attenuation.

Our desired signal is a $90$\unit{\hertz} sinusoidal. We do not want to lose
this signal, i.e., we want our transfer function $H(s)$ to approximately have a
unity gain at this frequency. We set $2\pi RC \leq \nicefrac{1}{900}$ or $RC
\leq 1.768 \times 10^{-4}$\unit{\ohm\farad} to satisfy this requirement. Using
some standard values of the resistance $R = 1$\unit{\kilo\ohm} and the
capacitance $C = 104$\unit{\nano\farad}, we obtain $RC = 1.04 \times 10^{-4}$,
achieving the desired goal. The attenuation at Teensy's PWM carrier frequency of
$36.6$\unit{\kilo\hertz} is found by \[ 20 \log_{10}\left\{\abs{H(j 2\pi
36600)}\right\} = -55.163\,\unit{\decibel}. \]

Lastly, we want to amplify the input voltage $v_i$ thrice in order to hit the
$0-10$\unit{\volt} mark. The gain of the non-inverting amplifier is $k = 1 +
\nicefrac{R_f}{R_{in}}$. We choose $R_f = 2$\unit{\kilo\ohm} and $R_{in} =
1$\unit{\kilo\ohm} to achieve $k = 3$. The high gain bandwidth product of the
\texttt{OP292} op-amp is read from its datasheet to be $\text{GBP} =
4$\unit{\mega\hertz}. Hence the transfer function from the input voltage $v_i$
to the output voltage $v_o$ that will be applied to the PI controller is
approximately given by \[ \frac{V_o(s)}{V_i(s)} =
\frac{k}{\frac{k}{2\pi\text{GBP}}s + 1} \approx \frac{3}{1.194\times 10^{-7}s +
1}, \] which will have a firm unity gain at our desired oscillation frequency of
$90$\unit{\hertz}.
