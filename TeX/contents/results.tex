\vspace{-1em}
\section{Results}
\label{sec:results}
\vspace{-1em}

We provide extensive simulation and experimental data and their interpretation,
supporting that the proposed analog signal generator works as intended.

\vspace{-1em}
\subsection{Simulation}
\vspace{-1em}

We perform a realistic LTSpice~\cite{ltspice} simulation of both second-order
filters derived in Sections~(\ref{ssec:second}, \ref{ssec:sallenkey}). One of
the important aspects of these designs is the selection of the op-amp. In order
to keep the common-mode voltage at $0$\unit{\volt} we choose the op-amps as CMOS
type. One such op-amp is \texttt{LM358}~\cite{lm358}, which is used in the final
design, even though a similar op-amp \texttt{OP292} was used in this simulation.

The PWM signal generated by Teensy~\cite{teensy} is simulated exactly with a
carrier frequency of $36.6$\unit{\kilo\hertz} modulating the signal \[
V_{\text{pwm}} = \nicefrac{3.3}{2} + \nicefrac{3.3}{2}\sin{(2\pi 90 t)}.\]
Finally, the impedance of the load (PI's controller) is read off from its
datasheet and inserted as a $100$\unit{\kilo\ohm} resistance. The circuit that
is simulated using LTSpice is presented in Figures~\ref{fig:real_sig_gen}
and~\ref{fig:sallenkey_sim} (the bottom plot).

\begin{figure}[htb] 
\includegraphics[width=8cm]{./figures/circuit.png}
\caption{The signal generator circuit in LTSpice} 
\label{fig:real_sig_gen}
\end{figure}

The simulation for the architecture in Section~\ref{ssec:second} generates the
relevant voltage responses, provided in Figure~\ref{fig:response}. The top plot
shows the PWM signal generated by Teensy modulating a sine-wave at
$90$\unit{\hertz} frequency. The individual plots in the middle show the output
of the first (cyan) and the second (purple) RC low-pass filters ($v_m$ and
$v_i$, respectivel) that extract the modulated signal from its PWM
representation. Finally, the last plot shows the thrice amplified signal through
the op-amp \texttt{OP292}.

\begin{figure}[t]
\includegraphics[width=0.5\textwidth]{./figures/pwm_filtered_one_two_final_signal.png}
\caption{The response from the simulation for one full period.} 
\label{fig:response}
\end{figure}


The performance of the Sallen-Key architecture from Section~\ref{ssec:sallenkey}
is shown in Figure~\ref{fig:sallenkey_sim} (top plot), where the values for the
resistances were taken to be $R_1 = R_2 = R = 68$\unit{\kilo\ohm}, $R_f
= 2R_{in} = 1$\unit{\kilo\ohm} and capacitances to be $C_1 = C_2 = C =
1$\unit{\nano\farad}. Even though the performance of the Sallen-Key filter of
Section~\ref{ssec:sallenkey} looks very similar to the $RC$+buffer filter of
Secton~\ref{ssec:second} in simulation, the experiments favor the Sallen-Key
significantly. We will implement this filter in our final design.

\begin{figure}[tbh]
\includegraphics[width=0.5\textwidth]{./figures/sallenkey.png}
\caption{The response of the Sallen-Key architecture.}
\label{fig:sallenkey_sim}
\end{figure}


\vspace{-1em}
\subsection{Experiment}
\vspace{-1em}

The Sallen-Key architecture is implemented on a simple setup on my tabletop.
A snapshot of Teensy's PWM signal that modulates the desired sine-wave output is
provided in Figure~\ref{fig:teensy_pwm}.


\begin{figure}[bh]
    \includegraphics[width=0.5\textwidth]{./figures/teensy_pwm_osc.jpg}
    \caption{A snapshot of Teensy's PWM signal with a carrier frequency of
    $36.6$\unit{\kilo\hertz}.}
    \label{fig:teensy_pwm}
\end{figure}

% \begin{figure}[t]
% \includegraphics[width=0.5\textwidth]{./figures/prototype.jpg}
% \caption{Prototype working on a LED} 
% \label{fig:exp}
% \end{figure}

Figure~\ref{fig:sallenkey_osc} shows the response of the second-order Sallen-Key
LPF observed through an oscilloscope. This is the output of the signal generator
in response to the Teensy generated $90$\unit{\hertz} $0-3.3$\unit{\volt} PWM
signal modulating the desired sine wave. This response is satisfactory and
should drive the PI controller without any issues.

\begin{figure}[bh]
    \includegraphics[width=0.5\textwidth]{./figures/output_osc.jpg}
    \caption{The response of the second-order Sallen-Key filter.}
    \label{fig:sallenkey_osc}
\end{figure}

We have also tested the signal generator by driving PI's piezoelectric actuator.
It turns out that since Teensy's imperfect PWM generation, the best values of
the various capacitances and resistances seen in the circuit
diagram~\ref{fig:final_circuit} are as given in Table~\ref{tab:expvalues}. These
are the values to be used in the final design.

{\renewcommand{\arraystretch}{1.5}
\begin{table}
    \centering
    \caption{Experimental values of resistances and capacitances.}
    \begin{tabular}{*6c}
        \toprule
        $R_1$ & $R_2$ & $C_1$ & $C_2$ & $R_f$ & $R_{in}$ \\    
        \hline
        \midrule
        $500$\unit{\kilo\ohm} & $100$\unit{\kilo\ohm} & $2.35$\unit{\nano\farad}
          & $1$\unit{\nano\farad} & $2$\unit{\kilo\ohm} &
        $1$\unit{\kilo\ohm} \\
        \bottomrule
    \end{tabular}
    \label{tab:expvalues}
    \vspace{-1em}
\end{table}
}
